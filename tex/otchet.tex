\documentclass[utf8x, 12pt]{G7-32} 
\begin{document} 
\setcounter{table}{0} 
\renewcommand{\thetable}{\arabic{table}} 
\renewcommand{\theequation}{\arabic{equation}} 
\begin{table} 
\centering 
\caption{Экспериментальные данные} 
\begin{tabular}{|c|c||c|c||c|c|} 
\hline 
№ & $x$ & № & $x$ & № & $x$ \\  
\hline 
1 & 23.00 & 8 & 20.00 & 15 & 20.00 \\ 
\hline 
2 & 22.00 & 9 & 23.00 & 16 & 21.00 \\ 
\hline 
3 & 23.00 & 10 & 23.00 & 17 & 24.00 \\ 
\hline 
4 & 19.00 & 11 & 21.00 & 18 & 20.00 \\ 
\hline 
5 & 22.00 & 12 & 22.00 & 19 & 23.00 \\ 
\hline 
6 & 23.00 & 13 & 23.00 & 20 & 22.00 \\ 
\hline 
7 & 23.00 & 14 & 20.00 & 21 & 22.00 \\ 
\hline 
\end{tabular} 
\end{table} 
Вычислим среднее арифметическое случайной величины $X$ по формуле (1): 
\begin{equation} 
\bar{x}=\frac{1}{n} \sum_{i=1}^n x_i =21.86 
\end{equation} 

Далее вычислим среднее квадратическое отклонение $S$  случайной величины $x$ по формуле (2): 
\begin{equation} 
\sigma_X=\sqrt{\frac{\sum_{i=1}^n\left(x_i-\bar{x}\right)^2}{n-1}} = 1.36 
\end{equation} 

Вычислим среднее квадратическое отклонение среднего арифмитического для случайной величины $x$ по формуле (3): 
\begin{equation} 
\sigma_{\bar{X}}=\frac{\sigma_X}{\sqrt{n}} = 0.23 
\end{equation} 

Доверительные границы неисключенной систематической погрешности $\sigma_{\theta}$ заданы паспортом термодатчика DS18B20: 
\begin{equation} 
\sigma_\theta=0.5^{\circ} \mathrm{C} 
\end{equation} 

Теперь вычислим суммаруню погрешность имзеряемой величины по формуле (5): 
\begin{equation} 
\Delta=\sqrt{\sigma_\theta^2+\sigma_{\bar{X}}^2} = 0.15 
\end{equation} 

В итоге, используя формулы (1) и (6), получим: 
\begin{equation} 
\mathrm{X}=\bar{X} \pm \Delta = 21.86\pm0.15 
\end{equation} 
\end{document}